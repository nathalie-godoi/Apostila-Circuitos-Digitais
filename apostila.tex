\documentclass[12pt]{article}

\usepackage[utf8]{inputenc}
\usepackage[brazil]{babel}
\usepackage{amsmath, amssymb}
\usepackage{geometry}
\usepackage{graphicx}
\usepackage{enumitem}
\usepackage{tikz}
\usetikzlibrary{intersections}
\usepackage[most]{tcolorbox}
\tcbset{
  mynote/.style={
    colback=gray!10,
    colframe=gray!80,
    fonttitle=\bfseries,
    coltitle=black,
    boxrule=0.5pt,
    arc=4pt,
    left=6pt,
    right=6pt,
    top=4pt,
    bottom=4pt
  }
}
\geometry{a4paper, margin=2.5cm}

\title{Apostila de Circuitos Digitais}
\author{Julia Acras, Isabella Stuart, Nathalie Godoi e Rafaela L.}
\date{XX/XX/2025}

\begin{document}

\maketitle

\tableofcontents

\section{Tabela Verdade}
A tabela verdade é um quadro que representa todas as possíveis combinações lógicas entre variáveis booleanas. Ela mostra como os valores de entrada (0 ou 1) influenciam a saída de um circuito.

Para construí-la, divide-se a tabela em duas partes: entradas (ou variáveis) e saídas (ou combinações resultantes). A cada nova variável adicionada, o número de combinações dobra, pois cada variável pode assumir dois valores: 0 ou 1.

Assim, para \( n \) variáveis, o número total de linhas na tabela será \( 2^n \). Por exemplo, com 7 variáveis, temos \( 2^7 = 128 \) combinações possíveis.

\textbf{Observação:} Neste material, utilizamos \( \overline{A} \) para representar a negação da variável \( A \).

\subsection*{Operações Lógicas}
\begin{itemize}
   \item \textbf{Conjunção (AND) :} \( A \cdot B \) - verdadeiro somente se \( A \) e \( B \) forem verdadeiros.
   \item \textbf{Disjunção (OR):} \( A + B \) - falso somente se \( A\) e \( B \) forem falsos.
   \item \textbf{Negação (NOT):} \( \overline{A} \) - inverte o valor lógico de \( A \).
   \item \textbf{Implicação:} \( A \rightarrow B \) - falso apenas se \( A \) for verdadeiro e \( B \) falso.
   \item \textbf{Bicondicional:} \( A \leftrightarrow B \) - verdadeiro se \( A \) e \( B \) tiverem o mesmo valor lógico.
\end{itemize}


\subsection*{Exemplo 1:}
\begin{center}
\begin{tabular}{|c|c|}
\hline
A & \( \overline{A} \) \\
\hline
1 & 0 \\
0 & 1 \\
\hline
\end{tabular}

\vspace{0.5em}
\textit{$\overline{A}$ tem sempre o valor lógico oposto de $A$.}
\end{center}

\subsection*{Exemplo 2:}
\begin{center}
\begin{tabular}{|c|c|c|}
\hline
A & B & \( A \cdot B \) \\
\hline
0 & 0 & 0 \\
0 & 1 & 0 \\
1 & 0 & 0 \\
1 & 1 & 1 \\ 
\hline
\end{tabular}

\vspace{0.5em}
\textit{A função só é verdadeira quando $A$ e $B$ são ambos $1$.}
\end{center}

\section{Álgebra de Boole}

A Álgebra de Boole utiliza apenas dois valores:
\begin{itemize}
   \item Verdadeiro: $1$
   \item Falso: $0$
\end{itemize}

As três operações fundamentais são:

\begin{enumerate}
   \item \textbf{AND} (E lógico) \\
   A saída da operação \textbf{AND} é verdadeira somente quando todas as entradas são verdadeiras, ou seja, basta que apenas uma entrada seja falsa (0) para tornar todo o circuito falso. 
   \item \textbf{OR} (OU lógico) \\
   A saída é verdadeira se ao menos uma das entradas é verdadeira. Logo, a única situação em que o circuito tem seu valor lógico falso é quando todas as entradas são 0.
   \item \textbf{NOT} (Negação) \\
   A saída é o inverso da entrada. Se a entrada for verdadeira (1), a saída será falsa (0) e vice-versa.
\end{enumerate} 

\begin{tcolorbox}[mynote, title= Observação: Relação com Diagrama de Venn]
As operações booleanas também podem ser representadas por diagramas de Venn:
\begin{itemize}
   \item \textbf{AND} – Interseção entre os conjuntos $A$ e $B$
   \item \textbf{OR} – União dos conjuntos $A$ e $B$
   \item \textbf{NOT} – Complemento do conjunto $A$
\end{itemize}
\end{tcolorbox}

\vspace{0.2cm}

\begin{center}
\textbf{Diagrama de Venn da expressão: $A + (B \cdot C)$}
\end{center}

\vspace{0.3cm}

\begin{center}
\begin{tikzpicture}
   \fill[red!30] (-1,0) circle (1.5); % Conjunto A

   \begin{scope}
   \clip (1,0) circle (1.5);
   \fill[blue!30] (0,1.8) circle (1.5);
\end{scope}

   \draw (-1,0) circle (1.5) node[left] {$A$};
   \draw (1,0) circle (1.5) node[right] {$B$};
   \draw (0,1.8) circle (1.5) node[above] {$C$};

   \node at (0,-2.5) {\footnotesize Região destacada: $A \cup (B \cap C)$};
\end{tikzpicture}
\end{center}

\vspace{0.2cm}

\begin{center}
\textbf{Diagrama de Venn da expressão: $\overline{A \cdot B \cdot C}$}
\end{center}

\vspace{0.3cm}

\begin{center}
\begin{tikzpicture}

\draw[black, thick, fill=purple!20] (-2.8,-1.8) rectangle (2.8,3.5);
   \begin{scope}
      \clip (-1,0) circle (1.5);
      \clip (1,0) circle (1.5);
      \clip (0,1.8) circle (1.5);
      \fill[white] (-2.8,-1.8) rectangle (2.8,3.5);
   \end{scope}

   \draw[black, thick] (-1,0) circle (1.5) node[left] {$A$};
   \draw[black, thick] (1,0) circle (1.5) node[right] {$B$};
   \draw[black, thick] (0,1.8) circle (1.5) node[above] {$C$};
   \node at (-2, 2.8) {$U$};

\end{tikzpicture}
\end{center}

\vspace{0.5cm}
\begin{center}
\footnotesize Região destacada: $\overline{A \cap B \cap C}$
\end{center}

\section{Expressões, Axiomas e Simplificação}

Sendo \( X, Y\) e \( Z\) variáveis que assumem os valores lógicos 0 e 1, e utilizando os símbolos:
\begin{itemize}
   \item \( + \) para disjunção (OU lógico)
   \item \( \cdot \) para conjunção (E lógico)
   \item \( \overline{X} \) para negação de X
\end{itemize}

As propriedades fundamentais da Álgebra de Boole são representadas abaixo:

\begin{center}
\begin{tabular}{|l|l|l|}
\hline
\textbf{Propriedade} & \textbf{Versão OR} & \textbf{Versão AND} \\
\hline
1 - Identidade       & \( X + 0 = X \)                             & \( X \cdot 1 = X \) \\
2 - Elemento Nulo    & \( X + 1 = 1 \)                             & \( X \cdot 0 = 0 \) \\
3 - Idempotência     & \( X + X = X \)                             & \( X \cdot X = X \) \\
4 - Complemento      & \( X + \overline{X} = 1 \)                  & \( X \cdot \overline{X} = 0 \) \\
5 - Involução        & \( \overline{\overline{X}} = X \)           & \( \overline{\overline{X}} = X \) \\
6 - Comutativa       & \( X + Y = Y + X \)                         & \( X \cdot Y = Y \cdot X \) \\
7 - Associativa      & \( (X + Y) + Z = X + (Y + Z) \)             & \( (X \cdot Y) \cdot Z = X \cdot (Y \cdot Z) \) \\
8 - Distributiva     & \( X + (Y \cdot Z) = (X + Y) \cdot (X + Z) \) & \( X \cdot (Y + Z) = X \cdot Y + X \cdot Z \) \\
9 - Absorção 1       & \( X + X \cdot Y = X \)                     & \( X \cdot (X + Y) = X \) \\
10 - Absorção 2      & \( X + \overline{X} \cdot Y = X + Y \)      & \( X \cdot (\overline{X} + Y) = X \cdot Y \) \\
11 - Consenso        & \( X \cdot Y + \overline{X} \cdot Z + Y \cdot Z = X \cdot Y + \overline{X} \cdot Z \) 
                     & \( (X + Y)(\overline{X} + Z)(Y + Z) = (X + Y)(\overline{X} + Z) \) \\
12 - De Morgan       & \( \overline{X + Y} = \overline{X} \cdot \overline{Y} \) 
                     & \( \overline{X \cdot Y} = \overline{X} + \overline{Y} \) \\
\hline
\end{tabular}
\end{center}

As propriedades 2, 8, 9 e 11 serão demonstradas com tabelas verdades. As outras ficam a cargo do leitor demonstar.

\section{Demonstração das Propriedades}

\subsection{Propriedade 2: Elemento Nulo}

Versão OR:
\begin{center}
\begin{tabular}{|c|c|}
\hline
$X$  & $X + 1$ \\
\hline
0 & 1 \\
1 & 1 \\
\hline
\end{tabular}
\end{center}

Assim, independente do valor lógico de $X$ a função sempre 1.

Versão AND: 
\begin{center}
\begin{tabular}{|c|c|}
\hline
$X$  & $X \cdot 0$ \\
\hline
0 & 0 \\
1 & 0 \\
\hline
\end{tabular}
\end{center}

Dessa forma, fica perceptível que o valor lógico de $X$ na função não altera o valor da função, que será sempre 0.

\subsection{Propriedade 8: Distributiva}

Versão OR:
\begin{center}
\begin{tabular}{|c|c|c|c|c|}
\hline
$X$  & $Y$ & $Z$ & $X + (Y \cdot Z)$ & $(X + Y) \cdot (X + Z)$ \\
\hline
0 & 0 & 0 & 0 & 0 \\
0 & 0 & 1 & 0 & 0 \\
0 & 1 & 0 & 0 & 0 \\
0 & 1 & 1 & 1 & 1 \\
1 & 0 & 0 & 1 & 1 \\
1 & 0 & 1 & 1 & 1 \\
1 & 1 & 0 & 1 & 1 \\
1 & 1 & 1 & 1 & 1 \\
\hline 
\end{tabular}
\end{center}


Versão AND:
\begin{center}
\begin{tabular}{|c|c|c|c|c|}
\hline
$X$  & $Y$ & $Z$ & $X \cdot (Y + Z)$ & $(X \cdot Y) + (X \cdot Z)$ \\
\hline
0 & 0 & 0 & 0 & 0 \\
0 & 0 & 1 & 0 & 0 \\
0 & 1 & 0 & 0 & 0 \\
0 & 1 & 1 & 0 & 0 \\
1 & 0 & 0 & 0 & 0 \\
1 & 0 & 1 & 1 & 1 \\
1 & 1 & 0 & 1 & 1 \\
1 & 1 & 1 & 1 & 1 \\
\hline
\end{tabular}
\end{center}

\subsection{Propriedade 9: Absorção 1}

Versão OR:
\begin{center}
\begin{tabular}{|c|c|c|}
\hline
$X$  & $Y$ & $X + (X \cdot  Y)$ \\
\hline
0 & 0 & 0 \\
0 & 1 & 0 \\
1 & 0 & 1 \\
1 & 1 & 1 \\
\hline
\end{tabular}
\end{center}

Versão AND:
\begin{center}
\begin{tabular}{|c|c|c|}
\hline
$X$  & $Y$ & $X \cdot (X + Y)$ \\
\hline
0 & 0 & 0 \\
0 & 1 & 0 \\
1 & 0 & 1 \\
1 & 1 & 1 \\
\hline
\end{tabular}
\end{center}

\subsection{Propriedade 11: Consensus}

Versão OR:
\begin{center}
\begin{tabular}{|c|c|c|c|c|}
\hline
$X$  & $Y$ & $Z$ & $X \cdot Y + \overline{X} \cdot Z + Y \cdot Z$ & $X \cdot Y + \overline{X} \cdot Z$ \\
\hline
0 & 0 & 0 & 0 & 0 \\
0 & 0 & 1 & 1 & 1 \\
0 & 1 & 0 & 0 & 0 \\
0 & 1 & 1 & 1 & 1 \\
1 & 0 & 0 & 0 & 0 \\
1 & 0 & 1 & 0 & 0 \\
1 & 1 & 0 & 1 & 1 \\
1 & 1 & 1 & 1 & 1 \\
\hline
\end{tabular}
\end{center}

Versão AND:
\begin{center}
\begin{tabular}{|c|c|c|c|c|}
\hline
$X$  & $Y$ & $Z$ &  $(X + Y)(\overline{X} + Z)(Y + Z)$ & $(X + Y)(\overline{X} + Z)$ \\
\hline
0 & 0 & 0 & 0 & 0 \\
0 & 0 & 1 & 0 & 0 \\
0 & 1 & 0 & 1 & 1 \\
0 & 1 & 1 & 1 & 1 \\
1 & 0 & 0 & 0 & 0 \\
1 & 0 & 1 & 1 & 1 \\
1 & 1 & 0 & 1 & 1 \\
1 & 1 & 1 & 1 & 1 \\
\hline
\end{tabular}
\end{center}

\section{MINTERMOS E MAXTERMOS}
Mintermos e maxternos são formas canônicas de representar funções booleanas
Toda expressão booleanas pode ser obtida através da sua tabela verdade
\begin{quote}
\textbf{Observação:} Dizemos que uma expressão está na \textit{forma canônica} quando ela é representada por uma \textbf{soma de mintermos} (forma canônica disjuntiva) ou um \textbf{produto de maxtermos} (forma canônica conjuntiva). Em ambas, todas as variáveis da função aparecem em cada termo, direta ou negada.
\end{quote}

\subsection{Mintermos}
Mintermi: é a forma canônica de representar uma linha da tabela verdade utilizando o operador AND (E lógico) entre todas as variáveis da função.
Cada variável pode aparecer de duas formas: direta ou negada, dependendo do valor que ela assume na linha da tabela.

\textit{Notação}: 
\begin{itemize}
   \item Valor 0: a variável aparece \textbf{negada} (exemplo: $A = 0 \Rightarrow \overline{A}$)
   \item Valor 1: a variável aparece \textbf{direta} (exemplo: $A = 1 \Rightarrow A$)
\end{itemize}

O mintermo da primeira linha da tabela verdade é o $m_0$ e o da útima será $m_{2^n - 1}$, sendo $n$ o número de variáveis da função.
Para construir um mintermo, pega-se todas as variáveis da linha e usa-se a operação lógica \textbf{AND} entre elas, respeitando o uso da forma direta ou negada de acordo com os valores. Assim, cada mintermo representa exatamente uma linha onde a saída da função é igual a 1.

\begin{tcolorbox}[mynote, title=Notação – Forma Canônica por Mintermos]
A notação \textit{$f(A, B, C) = \sum m(x, y, z)$} indica que a função $f$ é expressa como uma \textbf{soma de mintermos}, ou seja, uma disjunção (OU) dos termos onde a saída é igual a 1 na tabela verdade. As letras "x", "y" e "z" indicam as \textbf{linhas} que a função é \textbf{1}. 
\end{tcolorbox} 

\vspace{0.5cm}

\subsection{Maxtermo}
Maxtermo: é a forma canônica de representar uma linha da tabela verdade utilizando o operador OR (OU lógico) entre todas as variáveis da função. 
A contrução dos maxtermos é semlhante à dos mintemos, porém segue a lógica inversa:

\textit{Notação}:
\begin{itemize}
   \item Valor 0: a variável aparece \textbf{direta} (exemplo: $A = 0 \Rightarrow A$)
   \item Valor 1: a variável aparece \textbf{negada} (exemplo: $A = 1 \Rightarrow \overline{A}$)
\end{itemize}

Para cada linha da tabela verdade em que a saída é 0, pode-se formar um maxtermo. O maxtermo da primeira linha da tabela verdade é o $M_0$, e o último será $M_{2^n - 1}$, com $n$ sendo o número de variáveis.

\begin{tcolorbox}[mynote, title=Notação – Forma Canônica por Maxtermos]
A notação \textit{$f(A, B, C) = \prod M(x, y, z)$} indica que a função $f$ é expressa como uma \textbf{produto de maxtermos}, ou seja, uma conjunção (AND) dos termos onde a saída é igual a 0 na tabela verdade. As letras "x", "y" e "z" indicam as \textbf{linhas} que a função é \textbf{zero}.
\end{tcolorbox} 

\vspace{0.5cm}

\textbf{Exemplo}:
\[
f = A + B
\]

\begin{center}
\begin{tabular}{|c|c|c|c|c|}
\hline
$A$  & $B$ & $f$ & Mintermo & Maxtermo \\
\hline
0 & 0 & 0 & $m_0 = \overline{A}\overline{B}$ & $M_0 = A + B$ \\
0 & 1 & 1 & $m_1 = \overline{A}B$            & $M_1 = A + \overline{B}$ \\
1 & 0 & 1 & $m_2 = A\overline{B}$            & $M_2 = \overline{A} + B$ \\
1 & 1 & 1 & $m_3 = AB$                       & $M_3 = \overline{A} + \overline{B}$ \\
\hline
\end{tabular}
\end{center}

\begin{itemize}
   \item Os \textbf{mintermos} representam as linhas em que $f = 1$.
   \item Os \textbf{maxtermos} representam as linhas em que $f = 0$.
   \item A função $f = A + B$ tem como forma canônica:
   \begin{itemize}
      \item \textbf{Soma de mintermos:} $f = m_1 + m_2 + m_3 = \overline{A}B + A\overline{B} + AB$
      \item \textbf{Produto de maxtermos:} $f = M_0 = (A + B)$
   \end{itemize}
\end{itemize}

\begin{tcolorbox}[mynote, title=Mintermos, Maxtermos e Complemento]
Como cada mintermo é o complemento de seu maxtermo correspondente, podemos dizer que o complemento de uma função $f$ é representado pela multiplicação (AND) dos maxtermos associados às linhas onde $f = 0$.

No exemplo acima, como $f = A + B$, temos que $f = 1$ nas linhas 1, 2 e 3, e $f = 0$ apenas na linha 0. Logo, o complemento de $f$ é:

\[
f' = M_0 = A + B
\quad \Rightarrow \quad f = \overline{A + B}
\]

Assim, mostramos que o maxtermo $M_0$ representa exatamente o valor de $f'$ nesta função.
\end{tcolorbox}

\section{Circuito Combinacional}

\textbf{Circuitos combinacionais} são aqueles que só dependem do estado atual da entrada, ou seja, não possuem \textbf{memória}.
Um exemplo simples é um circuito onde a saída $f$ é dada por $f = A \cdot B$, usando uma porta \textbf{AND} de duas entradas.

\vspace{0.5cm}

Esses circuitos incluem:

\begin{itemize}
   \item Somadores
   \item Comparadores
   \item Codificadores
   \item Decodificadores
   \item Multiplexadores
\end{itemize}

\section{Circuitos Sequenciais}

\textbf{Diferente} dos circuitos combinacionais, os \textbf{circuitos sequenciais} dependem tanto das \textbf{entradas atuais} como do \textbf{estado anterior (memória)}.

Eles utilizam elementos de \textbf{armazenamento}, como flip-flops, para guardar o estado interno. Portanto, a saída é uma função das entradas e do estado armazenado. 

\vspace{0.5cm}

São utilizados em:

\begin{itemize}
   \item Regustradores
   \item Máquinas de Estados
   \item Contadores
\end{itemize}

\vspace{0.5cm}

Os circuitos sequenciais podem ser \textbf{síncronos} (controlados por um sinal de clock) ou \textbf{assíncronos} (controlados por variações nas entradas)

\end{document}

